\hypertarget{was-man-einstellen-kann}{%
\section{Was man einstellen kann}\label{was-man-einstellen-kann}}

Einige \emph{Dinge} kann man bei der Diplomarbeit auch noch anpassen.
Einiges ist sogar in der lyx- bzw. tex-Datei schon vorgesehen -- bitte
den Anfang des Source-Codes aufmerksam lesen.

\hypertarget{inhaltkapitel}{%
\subsection{Inhalt/Kapitel}\label{inhaltkapitel}}

Einige Ideen zur Gliederung der Arbeit gibt es unter
\url{http://www.diplomarbeiten-bbs.at/} \footnote{\url{http://www.diplomarbeiten-bbs.at/erstellung/durchf\%C3\%BChrung/gliederung-der-diplomarbeit-und-formale-vorgaben-0}}

\hypertarget{einseitigzweiseitig}{%
\subsection{Einseitig/zweiseitig}\label{einseitigzweiseitig}}

Korrekturexemplar immer einseitig drucken. Das fertige Buch ist dann
beidseitig zu drucken.

Begründete Abweichungen sind bitte mit dem Betreuer abklären -- die
Vorlage ist dann anzupassen (Überschriften, Vorlage für Kaitel, Bundsteg
usw.).

\hypertarget{farben}{%
\subsection{Farben}\label{farben}}

\begin{itemize}
\tightlist
\item
  Deckblatt mit Logo immer in Farbe
\item
  Farbausdrucke sind viel teurer -- Notwendigkeit prüfen und mit dem
  Betreuer abklären.
\item
  am aufwändigsten: ein paar Seiten in Farbe
\item
  falls Schwarz-Weiß: kein Logo beim laufenden Text
\end{itemize}

\hypertarget{autor}{%
\subsection{Autor}\label{autor}}

Bei der Diplomarbeit muss jeder Text einem Autor zuzuordnen sein. Man
kann das wie in der Vorlage in der Fußzeile machen. Alternativ kann man
auch eine Übersicht als Anhang eingefügt werden -- bitte mit dem
Betreuer abklären.

\hypertarget{absuxe4tze}{%
\subsection{Absätze}\label{absuxe4tze}}

\hypertarget{einzug}{%
\subsubsection{Einzug}\label{einzug}}

Den Beginn eines neuen Absatzes kann man durch Abstand oder durch
Einrücken kennzeichnen.

Diese Einstellung wird in der Vorlage bzw. im Header gemacht:

\begin{Shaded}
\begin{Highlighting}[]
\CommentTok{%\textbackslash{}parindent0pt % auskommentieren, wenn keine Einrueckung der }
               \CommentTok{% ersten Absatzzeile gewuenscht}
\CommentTok{%\textbackslash{}parskip1.5ex plus0.5ex minus0.5ex % flexibler Absatzabstand}
\end{Highlighting}
\end{Shaded}

Die Option \texttt{parskip=half} bei \texttt{documentclass} ersetzt
bereits den Absatzeinzug durch einen Absatzabstand.

oder \url{http://ctan.org/pkg/parskip}

\begin{Shaded}
\begin{Highlighting}[]
\BuiltInTok{\textbackslash{}usepackage}\NormalTok{[parfill]\{}\ExtensionTok{parskip}\NormalTok{\}}
\end{Highlighting}
\end{Shaded}

\hypertarget{noch-schuxf6ner}{%
\subsubsection{noch schöner}\label{noch-schuxf6ner}}

\url{http://www.khirevich.com/latex/microtype/}

\hypertarget{aufzuxe4hlungen}{%
\subsection{Aufzählungen}\label{aufzuxe4hlungen}}

Man kann die Einrückung und vieles mehr anpassen:

\begin{Shaded}
\begin{Highlighting}[]
\BuiltInTok{\textbackslash{}usepackage}\NormalTok{\{}\ExtensionTok{enumitem}\NormalTok{\}}
\FunctionTok{\textbackslash{}setlist}\NormalTok{[1]\{labelindent=}\FunctionTok{\textbackslash{}parindent}\NormalTok{\}}
\FunctionTok{\textbackslash{}setlist}\NormalTok{\{align=left\}}
\FunctionTok{\textbackslash{}setlist}\NormalTok{[itemize]\{leftmargin=*\}}

\CommentTok{% oder: }
\FunctionTok{\textbackslash{}setlength\textbackslash{}partopsep}\NormalTok{\{0.5ex\}}
\end{Highlighting}
\end{Shaded}

\hypertarget{warnungen}{%
\subsection{Warnungen}\label{warnungen}}

\begin{Shaded}
\begin{Highlighting}[]
\FunctionTok{\textbackslash{}sloppy}  \CommentTok{% etwas laxere Abstandskontrolle (weniger Fehlermeldungen)}
\end{Highlighting}
\end{Shaded}

\hypertarget{listings}{%
\subsection{Listings}\label{listings}}

Sonderzeichen:

\url{http://en.wikibooks.org/wiki/LaTeX/Source_Code_Listings\#Encoding_issue}

Anpassen:

\url{http://stackoverflow.com/questions/1965702/how-to-mark-line-breaking-of-long-lines}

\url{http://www.bollchen.de/blog/2011/04/good-looking-line-breaks-with-the-listings-package/}

\hypertarget{zitierstil}{%
\subsection{Zitierstil:}\label{zitierstil}}

Bitte mit dem Betreuer abklären.

\begin{itemize}
\tightlist
\item
  plaindin - mit nummern {[}1{]}
\item
  alphadin - mit Text+Jahr {[}Hor99{]}
\end{itemize}

\hypertarget{glossar}{%
\subsection{Glossar}\label{glossar}}

\begin{itemize}
\tightlist
\item
  \url{http://texwelt.de/wissen/fragen/10496/glossaries-alle-symbole-nur-verwendete-abkurzungen-anzeigen}
\item
  \url{http://en.wikibooks.org/wiki/LaTeX/Glossary} für die
  verschiedenen Formen
\end{itemize}

\hypertarget{ausdruck-zu-weit-obenunten}{%
\subsection{Ausdruck zu weit
oben/unten}\label{ausdruck-zu-weit-obenunten}}

Man kann das Layout anpassen: Position auf Seite

\begin{Shaded}
\begin{Highlighting}[]
\FunctionTok{\textbackslash{}voffset}\NormalTok{10mm}
\end{Highlighting}
\end{Shaded}

\hypertarget{urls}{%
\subsection{URLs}\label{urls}}

Man kann/sollte das \texttt{hyperref}-Paket anpassen, die bunten Links
kann man ausschalten.

\begin{Shaded}
\begin{Highlighting}[]
\FunctionTok{\textbackslash{}hypersetup}\NormalTok{\{breaklinks=true,}
\NormalTok{bookmarks=true,}
\NormalTok{pdfauthor=\{Mein Name\},}\CommentTok{% <------------------- anpassen!}
\NormalTok{pdftitle=\{Die Diplomarbeit\},}\CommentTok{% <------------------- anpassen!}
\NormalTok{colorlinks=true,}
\NormalTok{citecolor=blue,}
\NormalTok{urlcolor=blue,}
\NormalTok{linkcolor=magenta,}
\NormalTok{pdfborder=\{0 0 0\}\}}

\FunctionTok{\textbackslash{}urlstyle}\NormalTok{\{same\}}
\end{Highlighting}
\end{Shaded}

\hypertarget{pandoc-caption-mit-verweis}{%
\subsection{Pandoc-Caption mit
Verweis}\label{pandoc-caption-mit-verweis}}

siehe \url{https://github.com/chiakaivalya/thesis-markdown-pandoc}

This is how you insert figures using markdown. Also how to insert
citations copied over from your bibliography manager (I specifically
used Pandoc Citations in Papers).

\begin{verbatim}
![Figure from Walczak, 2010[@Walczak:2010uk]. 
    \label{mitosis} ](figures/mitosis_Walczak.pdf)
\end{verbatim}

\hypertarget{seitennummern}{%
\subsection{Seitennummern}\label{seitennummern}}

\url{http://www.golatex.de/wiki/\%5Cfrontmatter}

\begin{Shaded}
\begin{Highlighting}[]
\FunctionTok{\textbackslash{}frontmatter} \CommentTok{% switches to roman numbering}
\FunctionTok{\textbackslash{}mainmatter}
\FunctionTok{\textbackslash{}backmatter}
\end{Highlighting}
\end{Shaded}

\hypertarget{vorlagen}{%
\subsection{Vorlagen}\label{vorlagen}}

\hypertarget{tu-wien-informatik}{%
\subsubsection{TU Wien -- Informatik}\label{tu-wien-informatik}}

\begin{itemize}
\tightlist
\item
  \url{https://gitlab.cg.tuwien.ac.at/auzinger/vutinfth.git}

  \begin{itemize}
  \tightlist
  \item
    super Vorlage und Build-Skripts (auch für Windows)
  \end{itemize}
\item
  \url{http://www.informatik.tuwien.ac.at/dekanat/abschluss-master}
\item
  \url{http://www.informatik.tuwien.ac.at/fakultaet/informatik-code-of-ethics.pdf}
\item
  Alte Vorlage

  \begin{itemize}
  \tightlist
  \item
    \url{http://ieg.ifs.tuwien.ac.at/~aigner/download/tuwien.sty}
  \item
    von dort sind die Abkürzungen kopiert
  \end{itemize}
\end{itemize}

\hypertarget{andere}{%
\subsubsection{Andere}\label{andere}}

\begin{itemize}
\tightlist
\item
  \url{https://www.ctan.org/tex-archive/macros/latex/contrib/etdipa?lang=de}
\item
  \url{https://github.com/Digital-Media/HagenbergThesis}

  \begin{itemize}
  \tightlist
  \item
    enthält viele Tipps
  \end{itemize}
\item
  \url{https://github.com/philipmichel/vorlage-diplomarbeit-latex/blob/master/diplom.pdf}
\end{itemize}

\hypertarget{tipps-zur-diplomarbeit}{%
\section{Tipps zur Diplomarbeit}\label{tipps-zur-diplomarbeit}}

\hypertarget{source}{%
\subsection{Source}\label{source}}

\begin{itemize}
\tightlist
\item
  mit Java/Python/*-doc
\item
  sinnvolle Namen
\item
  Codierrichtlinien eingehalten

  \begin{itemize}
  \tightlist
  \item
    definiert?
  \item
    welche?
  \end{itemize}
\end{itemize}

\hypertarget{buch}{%
\subsection{Buch}\label{buch}}

\begin{itemize}
\item
  einheitliche Formatierung

  \begin{itemize}
  \tightlist
  \item
    Formatvorlage eingehalten
  \end{itemize}
\item
  Layout

  \begin{itemize}
  \tightlist
  \item
    kein halb-leeren Seiten
  \item
    wichtig bei Bildern und Listings
  \end{itemize}
\item
  Grammatik + Rechtschreibung
\item
  Irgendwie sollte man ein (einheitliches) Bild vom Leser haben:\newline
  Negaivbeispiel:

  \begin{itemize}
  \tightlist
  \item
    einerseits eine Dummy dem man \texttt{mdir} erklären muss
  \item
    andererseits kann er Python-Pakete installieren
  \end{itemize}
\end{itemize}

\hypertarget{umfang}{%
\subsubsection{Umfang}\label{umfang}}

\begin{itemize}
\tightlist
\item
  ohne Bilder aus dem Web bzw. Füllbilder
\end{itemize}

\hypertarget{formulierungen}{%
\subsection{Formulierungen}\label{formulierungen}}

NICHT:

\begin{itemize}
\tightlist
\item
  romanartige Erzählungen
\item
  seitenlange Installationsanweisungen statt Link auf Anleitung im Web
\end{itemize}

\hypertarget{listings-1}{%
\subsection{Listings}\label{listings-1}}

\begin{itemize}
\item
  wenig
\item
  ohne Kommentare

  \begin{itemize}
  \tightlist
  \item
    dafür mit Erklärung
  \item
    mit Verweis auf File
  \end{itemize}
\item
  UML Diagramme als Ergänzung
\item
  Anordung: Floating
\end{itemize}

\hypertarget{bilder}{%
\subsection{Bilder}\label{bilder}}

\begin{itemize}
\tightlist
\item
  Floating
\item
  Quellenangabe
\end{itemize}

\hypertarget{versionsverwaltung}{%
\subsection{Versionsverwaltung}\label{versionsverwaltung}}

\begin{itemize}
\tightlist
\item
  je unabhängigen Teilprojekt
\item
  sinnvolle Kommits (Nachrichten)
\item
  regelmäßig
\end{itemize}

\hypertarget{ppm}{%
\subsection{PPM}\label{ppm}}

\begin{itemize}
\tightlist
\item
  Entscheidung: Wieviel?
\item
  NICHT unreflektierte Analysen vom Projektstart im Anhang
\end{itemize}

\hypertarget{quellen}{%
\subsection{Quellen}\label{quellen}}

\begin{itemize}
\tightlist
\item
  alle Tools inkl. Versionsnummer
\item
  alle Zitate bzw. übernommenen Meinungen
\item
  Datenblätter \ua
\end{itemize}

\hypertarget{skripts}{%
\section{Skripts}\label{skripts}}

\hypertarget{pandoc-nach-latex}{%
\subsection{Pandoc nach Latex}\label{pandoc-nach-latex}}

\begin{Shaded}
\begin{Highlighting}[]
\CommentTok{#! /bin/bash}
\CommentTok{#set -x}
\CommentTok{#set -v}
\KeywordTok{set} \ExtensionTok{-e}

\VariableTok{PANDOCMODULES=}\NormalTok{markdown+auto_identifiers}
\VariableTok{PANDOCMODULES=$\{PANDOCMODULES\}}\NormalTok{+definition_lists}
\CommentTok{#PANDOCMODULES=$\{PANDOCMODULES\}+compact_definition_lists}
\VariableTok{PANDOCMODULES=$\{PANDOCMODULES\}}\NormalTok{+fenced_code_attributes}
\VariableTok{PANDOCMODULES=$\{PANDOCMODULES\}}\NormalTok{+autolink_bare_uris}
\VariableTok{PANDOCMODULES=$\{PANDOCMODULES\}}\NormalTok{+simple_tables+table_captions}
\VariableTok{PANDOCMODULES=$\{PANDOCMODULES\}}\NormalTok{+inline_notes+footnotes}

\VariableTok{PANDOCOPT=}\StringTok{"--listings-S -N -f }\VariableTok{$\{PANDOCMODULES\}}\StringTok{"}

\FunctionTok{mkdir}\NormalTok{ -p ../kaptex/}
\FunctionTok{rm}\NormalTok{ -f ../kaptex/*}

\KeywordTok{for} \ExtensionTok{f}\NormalTok{ in *.md}
\KeywordTok{do}
   \VariableTok{out=$(}\FunctionTok{basename} \VariableTok{$f}\NormalTok{ .md}\VariableTok{)}\NormalTok{.tex}
   \BuiltInTok{echo}\NormalTok{ -n }\VariableTok{$f} \StringTok{" "}
   \ExtensionTok{pandoc} \VariableTok{$\{PANDOCOPT\}} \VariableTok{$f}\NormalTok{ -o ../kaptex/}\VariableTok{$out}
\KeywordTok{done}
\end{Highlighting}
\end{Shaded}

\hypertarget{diplomarbeit-bauen}{%
\subsection{Diplomarbeit bauen}\label{diplomarbeit-bauen}}

Wichtig -- damit alle Seitenummern und Verweise passen:

\begin{itemize}
\tightlist
\item
  erster Latex-Lauf
\item
  \texttt{makeindex} und \texttt{bibref} aufrufen
\item
  noch zweimal Latex
\end{itemize}

\begin{Shaded}
\begin{Highlighting}[]
\BuiltInTok{cd}\NormalTok{ kapmd}
\ExtensionTok{./create.sh}
\BuiltInTok{cd}\NormalTok{ ..}
\ExtensionTok{pdflatex}\NormalTok{ diplbuch.tex }\KeywordTok{&&}
\ExtensionTok{makeindex}\NormalTok{ -c -q diplbuch.idx }\KeywordTok{&&}
\ExtensionTok{bibtex}\NormalTok{ diplbuch}
\ExtensionTok{pdflatex}\NormalTok{ diplbuch.tex }\KeywordTok{&&}
\ExtensionTok{pdflatex}\NormalTok{ diplbuch.tex}
\end{Highlighting}
\end{Shaded}

\hypertarget{tipps-von-der-uni-hagenberg}{%
\section{Tipps von der Uni
Hagenberg}\label{tipps-von-der-uni-hagenberg}}

Angepasste Kopie aus \citep{hagenberg}.

\hypertarget{drucken}{%
\subsection{Drucken}\label{drucken}}

Vor dem Drucken der Arbeit empfiehlt es sich, einige Dinge zu beachten,
um unnötigen Aufwand (und auch Kosten) zu vermeiden.

\#\#\#Drucker und Papier

Die Abschlussarbeit sollte in der Endfassung unbedingt auf einem
qualitativ hochwertigen Laserdrucker ausgedruckt werden, Ausdrucke mit
Tintenstrahldruckern sind \emph{nicht} ausreichend. Auch das verwendete
Papier sollte von guter Qualität (holzfrei) und üblicher Stärke
(mind.~\(80\; {\mathrm g} / {\mathrm m}^2\)) sein. Falls \emph{farbige}
Seiten notwendig sind, sollte man diese einzeln\footnote{Tip: Mit
  \emph{Adobe Acrobat} lassen sich sehr einfach einzelne Seiten des
  Dokuments für den Farbdruck auswählen und zusammenstellen.} auf einem
Farb-Laserdrucker ausdrucken und dem Dokument beifügen.

Übrigens sollten \textbf{alle} abzugebenden Exemplare \textbf{gedruckt}
(und nicht kopiert) werden! Die Kosten für den Druck sind heute nicht
höher als die für Kopien, der Qualitätsunterschied ist jedoch -- \va~bei
Bildern und Grafiken -- meist deutlich.

\#\#\#Druckgröße

Zunächst sollte sichergestellt werden, dass die in der fertigen
PDF-Datei eingestellte Papiergröße tatsächlich \textbf{A4} ist! Das geht
\zB~mit \emph{Adobe Acrobat} oder \emph{SumatraPDF} über \texttt{File}
\(\rightarrow\) \texttt{Properties}, wo die Papiergröße des Dokuments
angezeigt wird:

\begin{center}
\textbf{Richtig:} A4 = $8{,}27 \times 11{,}69$ in \bzw\ $21{,}0 \times 29{,}7$ cm.
\end{center}

Falls das nicht stimmt, ist vermutlich irgendwo im Workflow
versehentlich \textbf{Letter} als Papierformat eingestellt.

Ein häufiger und leicht zu übersehender Fehler beim Ausdrucken von
PDF-Dokumenten wird durch die versehentliche Einstellung der Option
```Fit to page''' im Druckmenü verursacht, wobei die Seiten meist zu
klein ausgedruckt werden. Überprüfen Sie daher die Größe des Ausdrucks
anhand der eingestellten Zeilenlänge oder mithilfe einer Messgrafik, wie
am Ende dieses Dokuments gezeigt. Sicherheitshalber sollte diese
Messgrafik bis zur Fertigstellung der Arbeit beibehalten und die
entsprechende Seite erst ganz am Schluss zu entfernt werden. Wenn, wie
häufig der Fall, einzelne Seiten getrennt in Farbe gedruckt werden, so
sollten natürlich auch diese genau auf die Einhaltung der Druckgröße
kontrolliert werden!

\hypertarget{schlussbemerkungen}{%
\subsection{Schlussbemerkungen}\label{schlussbemerkungen}}

An dieser Stelle sollte eine Zusammenfassung der Abschlussarbeit stehen,
in der auch auf den Entstehungsprozess, persönliche Erfahrungen,
Probleme bei der Durchführung, Verbesserungsmöglichkeiten, mögliche
Erweiterungen \usw~eingegangen werden kann. War das Thema richtig
gewählt, was wurde konkret erreicht, welche Punkte blieben offen und wie
könnte von hier aus weitergearbeitet werden?

\hypertarget{lesen-und-lesen-lassen}{%
\subsubsection{Lesen und lesen lassen}\label{lesen-und-lesen-lassen}}

Wenn die Arbeit fertig ist, sollten Sie diese zunächst selbst nochmals
vollständig und sorgfältig durchlesen, auch wenn man vielleicht das
mühsam entstandene Produkt längst nicht mehr sehen möchte. Zusätzlich
ist sehr zu empfehlen, auch einer weiteren Person diese Arbeit anzutun
-- man wird erstaunt sein, wie viele Fehler man selbst überlesen hat.

\hypertarget{checkliste}{%
\subsubsection{Checkliste}\label{checkliste}}

Abschließend noch eine kurze Liste der wichtigsten Punkte, an denen
erfahrungsgemäß die häufigsten Fehler auftreten.

Diese Punkte bilden auch die Grundlage der routine-mäßigen
Formbegutachtung in Hagenberg.

\begin{itemize}
\tightlist
\item
  \textbf{Titelseite:} Länge des Titels (Zeilenumbrüche), Name,
  Studiengang, Datum.
\item
  \textbf{Erklärung:} vollständig Unterschrift.
\item
  \textbf{Inhaltsverzeichnis:} balancierte Struktur, Tiefe, Länge der
  Überschriften.
\item
  \textbf{Kurzfassung/Abstract:} präzise Zusammenfassung, passende
  Länge, gleiche Inhalte und Struktur.
\item
  \textbf{Überschriften:} Länge, Stil, Aussagekraft.
\item
  \textbf{Typographie:} sauberes Schriftbild, keine \emph{manuellen}
  Abstände zwischen Absätzen oder Einrückungen, keine überlangen Zeilen,
  Hervorhebungen, Schriftgröße, Platzierung von Fußnoten.
\item
  \textbf{Interpunktion:} Binde- und Gedankenstriche richtig gesetzt,
  Abstände nach Punkten (\va~nach Abkürzungen).
\item
  \textbf{Abbildungen:} Qualität der Grafiken und Bilder, Schriftgröße
  und -typ in Abbildungen, Platzierung von Abbildungen und Tabellen,
  Captions. Sind \emph{alle} Abbildungen (und Tabellen) im Text
  referenziert?
\item
  \textbf{Gleichungen/Formeln:} mathem.~Elemente auch im Fließtext
  richtig gesetzt, explizite Gleichungen richtig verwendet, Verwendung
  von mathem.~Symbolen.
\item
  \textbf{Quellenangaben:} Zitate richtig referenziert, Seiten- oder
  Kapitelangaben.
\item
  \textbf{Literaturverzeichnis:} mehrfach zitierte Quellen nur einmal
  angeführt, Art der Publikation muss in jedem Fall klar sein,
  konsistente Einträge, Online-Quellen (URLs) sauber angeführt.
\item
  \textbf{Sonstiges:} ungültige Querverweise (\textbf{??}), Anhang,
  Papiergröße der PDF-Datei (A4 = \(8.27 \times 11.69\) Zoll),
  Druckgröße und -qualität.
\end{itemize}
